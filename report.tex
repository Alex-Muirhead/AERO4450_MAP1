\documentclass[a4paper]{article}

\usepackage{reportFormat}
\addbibresource{sample.bib}

\title{
	\Large {\sc Mech}4470 Hypersonics and Rarefied Gas Dynamics\\
	\Huge Assignment 1
}

\author{Muirhead, Alex \dots \texttt{s4435062}}

\date{\today}

\begin{document}

\pagenumbering{gobble}
\maketitle

% \tableofcontents

\vspace{10em}

\newpage
\pagenumbering{arabic}

\section*{Question 1}

\subsection*{Part A}
The coefficients of lift and drag for the projectile can be calculated from the coefficients of pressure for each individual face, which is as follows.
\begin{equation*}
	C_{p,i} = \frac{p_i - p_\infty}{q_\infty}
\end{equation*}
Under the assumption of ideal gas, the dynamic pressure \(q = {\frac{1}{2}\rho v^2}\) can be expressed as \(q = p\frac{\gamma}{2}M^2\). Then the coefficient of pressure is reduced to:
\begin{equation*}
	C_{p,i} = \frac{2}{\gamma M_\infty^2} \left( \frac{p_i}{p_\infty} - 1 \right)
\end{equation*}
This simplifies calculations in the problem, as only the static pressure ratios across each subsequent face are required.

\begin{enumerate}[label=\arabic*)]
	\item There is no predicted shock expansion along face 1. With the assumption of inviscid flow, the pressure on this face is the same as free-stream static pressure.
	\begin{equation*}
		M_1 = M_\infty = 6,\
		\frac{p_1}{p_\infty} = 1
	\end{equation*}
	This gives \(C_{p1} = 0\).

	\item The flow passes through a \ang{10} convex turn, and is expanded through Prandtl-Meyer waves. Note that, due to shock expansion being isentropic, the process conserves total pressure and \(p_{t1} = p_{t2}\).
	\begin{align*}
		\Delta \omega = \ang{10} &= \nu(M_2) - \nu(M_1) \\
		\nu(M_2) &= \nu(M_1) + \ang{10} \\
		&= \ang{84.95} + \ang{10} = \ang{94.95} \\
		M &\approx 7.85
	\end{align*}
	\begin{equation*}
		\frac{p_2}{p_\infty} = \frac{p_2}{p_{t2}} \left( \frac{p_1}{p_{t1}} \right)^{-1} \frac{p_1}{p_\infty}
		= \frac{\num{1.158e-4}}{\num{6.334e-4}} \approx 0.1828
	\end{equation*}
	This gives \(C_{p2} = -0.032\).

	\item The flow passes through a \ang{10} concave turn, and is compressed through an oblique shockwave. At Mach 7.85, the shockwave angle is \(\theta \approx \ang{16}\) and \(M_{n2} = M_2 \sin{16} \approx 2.16\). From \cite{NASATables}, this gives
	\begin{gather*}
		M_{n3} = 0.553 \\
		M_3 = \frac{M_{n3}}{\sin(\theta - \delta)} \approx 5.10
	\end{gather*}
	\begin{align*}
		\frac{p_3}{p_\infty} &= \frac{p_3}{p_2} \frac{p_2}{p_\infty} = 5.277 \times 0.1828 \approx 0.9648
	\end{align*}
	This gives \(C_{p3} = \num{-1.397e-3}\).
\end{enumerate}

The flow on the lower faces (\(1-4\)) is calculated below, starting from the freestream values. As with before, the total pressure will be propagated through calculations.
\begin{enumerate}[label=\arabic*), start=4]
	\item The flow passes through a \(\delta = \ang{20}\) concave turn, and is compressed through an oblique shockwave.
	At Mach 6, the shockwave angle is \(\theta = \ang{28.5}\) and\(M_{n\infty} = M_{\infty}\sin(28.5) \approx 2.86\). From \cite{NASATables}, this gives
	\begin{gather*}
		M_{n4} = 0.484 \\
		M_4 = \frac{M_{n4}}{\sin(\theta - \delta)} \approx 3.27 \\
		\frac{p_4}{p_\infty} = 9.376
	\end{gather*}
	This gives \(C_{p4} = 0.332\).

	\item The flow passes through a \ang{10} convex turn, and is expanded through Prandtl-Meyer waves.
	\begin{align*}
		\Delta \omega = \ang{10} &= \nu(M_5) - \nu(M_4) \\
		\nu(M_5) &= \nu(M_4) + \ang{10} \\
		&= \ang{54.703} + \ang{10} = \ang{64.703} \\
		M_5 &\approx 3.92,\
	\end{align*}
	\begin{equation*}
		\frac{p_5}{p_\infty} = \frac{p_5}{p_{t5}} \left( \frac{p_4}{p_{t4}} \right)^{-1} \frac{p_4}{p_\infty}
		= \frac{\num{7.332e-3}}{\num{1.826e-2}} \times 9.376
		\approx 3.7645
	\end{equation*}
	This gives \(C_{p5} = 0.110\).

	\item The flow passes through a \(\delta = \ang{10}\) concave turn, and is compressed through an oblique shockwave. At Mach 3.92, the shockwave angle is \(\theta \approx \ang{22.5}\) and \(M_{n5} = M_5 \sin{22.5} \approx 1.5\). From \cite{NASATables}, this gives
	\begin{gather*}
		M_{n6} = 0.701 \\
		M_6 = \frac{M_{n6}}{\sin(\theta - \delta)} \approx 3.24
	\end{gather*}
	\begin{equation*}
		\frac{p_6}{p_\infty} = \frac{p_6}{p_5} \frac{p_5}{p_\infty}
		= 2.458 \times 3.7645 \approx 9.2531
	\end{equation*}
	This gives \(C_{p6} = 0.327\)

\end{enumerate}

The pressure on the rear of the projectile (face 7) is given as the static pressure of the freestream flow, and thus does not contribute any force to the projectile.
Overall coefficients of lift and drag can be calculated for the projectile from the individual coefficients of pressure as follows.
\begin{equation*}
	C_L = \sum_i c_{p,i} \frac{L\vec{n}_{\parallel i}}{c}, \quad
	C_D = \sum_i c_{p,i} \frac{L\vec{n}_{\perp i}}{c}
\end{equation*}
where \(L\) is the length of the face, \(\vec{n}\) the vector orthonormal to the face, and \(c\) the chord length of the projectile. Calculating this gives values of \(C_L \approx 0.227\) and \(C_D \approx 0.0695\).

\subsection*{Part B}
The free-stream conditions at the geometric altitude of 90,000 ft (approximately \SI{27432}{\metre}) are obtained from interpolating the values from Table~A6 of \cite{white2011fluid}.
The stagnation (or total) values are also calculated through isentropic relations, based on the projectile speed of Mach 6.
\begin{alignat*}{2}
	T_\infty    &= \SI{223.93}{\kelvin} &\quad
	T_{t\infty} &= \SI{1836.23}{\kelvin} \\
	p_\infty    &= \SI{1.78}{\kilo\pascal} &\quad
	p_{t\infty} &= \SI{2810.4}{\kilo\pascal}
\end{alignat*}

The lift and drag forces on the projectile can be calculated from the coefficients of lift and drag calculated in Part A.
\begin{equation*}
	L = \frac{\gamma}{2}p_\infty M_\infty^2 A C_L, \quad
	D = \frac{\gamma}{2}p_\infty M_\infty^2 A C_D
\end{equation*}
As only the cross-section of the projectile is presented, it is assumed this is constant, and thus the lift and drag forces are given per unit metre of span.
\begin{equation*}
	\frac{L}{b} = \SI{2.043e3}{\newton\per\metre},
	\quad
	\frac{D}{b} = \SI{6.247e2}{\newton\per\metre}
\end{equation*}

\subsection*{Part C}

Under ideal (inviscid) conditions, each shock-wave and Prandtl-Meyer wave is an adiabatic process, and as such the total temperature does not change across the projectile.
This leads to the temperature on each face being dependant on only the Mach number (under the assumption of a calorically perfect gas). Face 6 has the lowest Mach number of 3.24 . From \cite{NASATables} the corresponding temperature is \( 0.3226 \times T_t \approx \SI{610.73}{\kelvin}\) .

\subsection*{Part D}

From \cite{NASATables}, there is no solution for an oblique shockwave at Mach 6 past a turn angle of \ang{43}. Considering the half-angle of the projectile leading edge is \ang{10}, an angle of attack exceeding \ang{33} would cause a separated shock.

\pagebreak
\section*{Question 2}

\subsection*{Part A}

Under the initial assumption that the gas has the properties of dry air, the Mach number of the flow can be calculated as
\begin{equation}
	M = \frac{v}{\sqrt{\gamma R T}}
	= \frac{\SI{6300}{\metre\per\second}}{\SI{664.82}{\metre\per\second}}
	= 9.48
\end{equation}
where \(R = \SI{287}{\joule\per\kilogram\per\kelvin}\) and \(\gamma = 1.4\). From \cite{NASATables}, flow at Mach 9.476 will have a ratio between pitot and static pressure of \(\sfrac{p_{t2}}{p_1} = 116.18\). From the given static pressure of \(\SI{400}{\pascal}\), the expected pitot pressure is \(\SI{46.473}{\kilo\pascal}\).

\subsection*{Part B}

The tables of \cite{NASATables} do not contain tabulated relations for gases other than standard air. However, the Rayleigh pitot formula can be used. The new Mach number for the flow is given as
\begin{equation*}
	M = \frac{v}{\sqrt{\gamma R T}}
	= \frac{\SI{6300}{\metre\per\second}}{\SI{617.52}{\metre\per\second}}
	= 10.20
\end{equation*}
where \(R = \frac{R_u}{MW} = \SI{208.13}{\joule\per\kilogram\per\kelvin}\) and \(\gamma = \frac{5}{3} \approx 1.667\). Substituting these values into the Rayleigh pitot equation gives
\begin{equation*}
	\frac{p_{t2}}{p_1}
	= \left[ \frac{(\gamma+1)M^2}{2} \right]^\frac{\gamma}{\gamma-1}
	\left[ \frac{\gamma+1}{2\gamma M^2 - (\gamma-1)} \right]^\frac{1}{\gamma-1}
	\approx 153.32
\end{equation*}
Using the given static pressure of \(\SI{400}{\pascal}\), the expected pitot pressure with Argon as the test gas is \(\SI{61.329}{\kilo\pascal}\).

\newpage
\subsection*{Part C}

For the test gas \(CO_2\), there are no tabulated values within \cite{NASATables}. With the Rayleigh pitot equation having no analytical solution, a newtonian root finding algorithm is used to find the required Mach number.

\begin{minted}[tabsize=4,obeytabs,fontsize=\footnotesize,frame=lines,framesep=0.5em,linenos]{python}
from math import sqrt
from scipy.optimize import newton

gamma, R = 1.28, 188.92  # Values for Carbon Dioxide
T = 2000                 # Freestream temperature [K]

def RayleighPitot(Mach):
	return (
		pow( ((gamma+1)*Mach**2) / 2, gamma/(gamma-1) )
		* pow( (gamma+1) / (2*gamma*Mach**2 - (gamma-1)), 1/(gamma-1) )
	)

# Solve for pitot to static pressure ratio of 200:1
# Initial guess of Mach 10
Mach = newton(lambda M: RayleighPitot(M) - 200/1, 10)
a    = sqrt(gamma*R*T)

print(f"flow of Mach {Mach:.3f}, and travelling at {a*Mach:.3f}m/s")
\end{minted}
This results in a "\texttt{flow of Mach 12.876, and travelling at 8954.152m/s}"
The ratio of temperatures across the shock is given by
\begin{equation*}
	\frac{T_2}{T_1}
	= \frac{
		\left[ 2 \gamma M_1^2 - (\gamma-1) \right]
		\left[ (\gamma-1)M_1^2 + 2 \right]
	}{(\gamma+1)^2 M_1^2}
	\approx 23.83
\end{equation*}
This gives a temperature immediately behind the normal part of the bow shock of roughly \(23.83\times\SI{2000}{\kelvin} \approx \SI{49084.84}{\kelvin}\).
At these massive temperature differences, the gas cannot be considered calorically perfect. Under real conditions, both the heat capacities and ratio of specific heats will vary.

\printbibliography

\end{document}
